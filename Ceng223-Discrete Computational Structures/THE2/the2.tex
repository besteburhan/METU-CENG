\documentclass[12pt]{article}
\usepackage[utf8]{inputenc}
\usepackage{float}
\usepackage{amsmath}


\usepackage[hmargin=3cm,vmargin=6.0cm]{geometry}
%\topmargin=0cm
\topmargin=-2cm
\addtolength{\textheight}{6.5cm}
\addtolength{\textwidth}{2.0cm}
%\setlength{\leftmargin}{-5cm}
\setlength{\oddsidemargin}{0.0cm}
\setlength{\evensidemargin}{0.0cm}

%misc libraries goes here


\begin{document}

\section*{Student Information } 
%Write your full name and id number between the colon and newline
%Put one empty space character after colon and before newline
Full Name : Beste Burhan \\
Id Number : 2171395 \\

% Write your answers below the section tags
\section*{Answer 1}


\subsection*{a.}

\begin{table}[H]

\small
\centering
\caption{ Membership Table for  $ A \cap B \subseteq (A \cup \overline{B} )\cap (\overline{A} \cup B)$ }
\label{table:question 1/a}
\begin{tabular}{|c c|c c|c|c|c|c|}	%% specify column number
\hline 	
			
					%% line draw
\textbf{$A$} & \textbf{$B$} & \textbf{$\overline{A}$} & \textbf{$\overline{B}$} & \textbf{$A \cap B$} & \textbf{$A \cup \overline{B}$} & \textbf{$\overline{A }\cup B$} & \textbf{$(A \cup \overline{B})\cap(\overline{A }\cup B)  $}\\
\hline 
\hline 
1 & 1 & 0 & 0 & 1 & 1 & 1 & 1\\			%% rows distinguished with &
1 & 0 & 0 & 1 & 0 & 1 & 0 & 0\\
0 & 1 & 1 & 0 & 0 & 0 & 1 & 0\\
0 & 0 & 1 & 1 & 0 & 1 & 1 & 1\\
\hline 

\end{tabular}

\end{table}

\subsection*{b.}

\begin{table}[H]

\small
\centering
\caption{ Membership Table for  $\overline{A} \cap \overline{B} \subseteq (A \cup \overline{B} )\cap (\overline{A} \cup B)$ }
\label{table:question 1/a}
\begin{tabular}{|c c|c c|c|c|c|c|}	%% specify column number
\hline 	
			
					%% line draw
\textbf{$A$} & \textbf{$B$} & \textbf{$\overline{A}$} & \textbf{$\overline{B}$} & \textbf{$\overline{A} \cap \overline{B}$} & \textbf{$A \cup \overline{B}$} & \textbf{$\overline{A }\cup B$} & \textbf{$(A \cup \overline{B})\cap(\overline{A }\cup B)  $}\\
\hline 
\hline 
1 & 1 & 0 & 0 & 0 & 1 & 1 & 1\\			%% rows distinguished with &
1 & 0 & 0 & 1 & 0 & 1 & 0 & 0\\
0 & 1 & 1 & 0 & 0 & 0 & 1 & 0\\
0 & 0 & 1 & 1 & 1 & 1 & 1 & 1\\
\hline 

\end{tabular}

\end{table}




\section*{Answer 2}

\quad \quad Suppose that $A \cap B =  \emptyset$ ,$ f^{-1} ((A \cap B) \times C ) = \emptyset$. Since $f$ is a bijection, there are no two elements with same image (so for $f^{-1}$). Therefore,

\quad $ f^{-1} (A \times C) \cap f^{-1} (B \times C)) = \emptyset = f^{-1}((A\cap B) \times C)$
\vspace{10px}

\quad Suppose that $t \in f^{-1} (A \times C) \cap f^{-1} (B \times C) $,then $y \in f^{-1} (A \times C)$ and $y \in f^{-1}(B \times C)$
Hence, there exist $x_1$,$x_2$ such that $f^{-1} (\lbrace x_1,x_2 \rbrace) = y$ and there exist $x_3$,$x_4$ such that $f^{-1} (\lbrace x_3,x_4 \rbrace) = y$. Since f is a bijection, $x_1 = x_3$ , $x_2 = x_4$  and $x_1, x_3 \in A \cap B.$ $f^{-1} (\lbrace x_1,x_2 \rbrace) = y \in  f^{-1} ((A \cap B) \times C ) $. Therefore, $f^{-1} ((A \cap B) \times C ) \subseteq  f^{-1} (A \times C) \cap f^{-1} (B \times C)) $  and so\\$f^{-1} ((A \cap B) \times C ) =  f^{-1} (A \times C) \cap f^{-1} (B \times C)) $


\section*{Answer 3}
\subsection*{a.}
\quad Since $f(-2) = f(2) = ln9 $, f is not one-to-one.
$(-1) \in R $ but $ln(x^2 +5)$ can not be equal to $-1$ for any value of $x$, so f is not onto.

\subsection*{b.}
\quad To show that f is one to one , $f(x) =f(y) \rightarrow x=y$ should be shown
\begin{gather*} 
e^{e^{x^7}} = e^{e^{y^7}}\\
e^{x^7} = e^{y^7}\\
x^7=y^7\\
x=y
\end{gather*} so f is one to one.

\quad $(-1) \in R$ but $e^{e^{x^7}}$ can not be equal to $-1$. Therefore, f is not onto.


\section*{Answer 4}
\subsection*{a.}

\quad Since $A$ and $B$ are countable, $A \rightarrow N$ and $B \rightarrow N$ are injections. Therefore, there exist an injection $f: A \times B \rightarrow N^2 $

\quad if I take $g: N^2 \rightarrow N$ and  $g(x,y)=3^x.5^y$ , assume that $a,b,c,d \in N$ 
\begin{gather*} 
f(a,b) = f(c,d)\\
3^a.5^b = 3^c.5^d 
\end{gather*} if and only if when $ a=c ,b=d$, so g is an injection.\\
Therefore, $f\circ g : A \times B \rightarrow N$ is an injection.

\subsection*{b.}

\quad Assume that $B$ is countable.Because  $A \subseteq B$ and b is countable , I can list elements of $A$. It means that $A$ is countable, but it is not. There is a contradiction. Hence, $B$ is uncountable.

\subsection*{c.}

\quad There is an injection $f : B \rightarrow N$. Assume that $g : A \rightarrow B$, then $g$ is an injection. Since $f$ and $g$ are injections, then $f \circ g : A \rightarrow N$ is an injection.

\section*{Answer 5}



$ f_1(x) \leq Cf_2(x)$

Assume that $f_1(x) ,\quad f_2(x) =x$ as a increasing functions when $x>1$



\quad \quad \quad $0<f_1(x) \leq cx$

\subsection*{a.}

\quad $0 < ln(f_1(x)) \leq ln(cx) = lnc + lnx$ since lnc is a constant $ln(f_1(x))$ is $ \mathcal{O} (lnx)$

\subsection*{b.}

\quad $0< 3^{f_1(x)} \leq 3^{cx} \leq 3^c.3^x$ ($C=3^c$) so $ 3^{f_1(x)}$ is $\mathcal{O} (3^x) $ 
\section*{Answer 6}

\subsection*{a.}
\begin{gather*}
(3^x - 1) mod (3^y-1) = 3^{(x  mod y)}-1\\
(3^x-1-3^{x mod y} +1) mod (3^y-1)= 0\\
(3^x-3^{x mod y}) mod (3^y-1) =0\\
3^y-1 \mid (3^x-3^{x mod y})\\
x=ty+d \quad (for \quad x \quad mod \quad y)\\
3^y-1 \mid 3^{ty+d}-3^d\\
3^y-1 \mid 3^d(3^{ty}-1)\\
(3^y-1) \mid 3^d((3^y-1).(3^{t-1}+3^{t-2}+..1))
\end{gather*}
 
\subsection*{b.}

\begin{gather*} 
277 = 2.123 + 31\\
123 = 3.31 + 30\\
31 = 1.30 + 1\\
30 = 30.1  
\end{gather*} since 1 divides 30, gcd(277,123) = gcd(123,31) = gcd(31,30) = gcd(30,1) = 1.

\end{document}

​

